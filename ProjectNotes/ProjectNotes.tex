\documentclass[11pt]{article}
\usepackage[utf8]{inputenc}
\usepackage{pgfgantt}
\usepackage{rotating}

\title{GPU Accelerated Method for Constructing and Rendering Trees 
        \\ - \\ 
        Project Notebook}
\author{Thomas Mcloughlin - 100203952}
\date{02 / 10 / 2020}

\begin{document}
\maketitle

\pagebreak
\section*{Introduction}
This notebook will serve as a reference for all activities related to the completion
of my project. I will separate each entry by week, listing and describing everything
I did that week and how it will help my project progress.

\section*{Week 1 Sem 1, 28th Sept - 4th Oct}
This week I was introduced to the expectations for the project. I have started by 
scheduling a meeting with Stephen Laycock, my supervisor, and starting off my project 
proposal by doing some market research of similar software and reading some papers 
covering the subject of constructing trees using algorithms.

I have not yet decided any approach for the project due to being unsure what kind of 
aims I should be making, after my meeting with Stephen on the 5th I will hopefully 
know what route to be going down for the project.

\section*{Week 2 Sem 1, 5th Oct - 11th Oct}
After my meeting with Stephen on the 5th I am more aware of what direction I will be 
taking the project. I will be creating an OpenGL module that can be added into a 
project to allow for the construction of trees in whatever environment the user has 
created. There are many avenues of complexity I can go down but at the moment I have 
decided that my focus will be to have the branch creation be as realistic as possible 
and hopefully include some sort of obstacle avoidance system wherever the tree grows.
I have scheduled another meeting with Stephen for the same time next week on the 12th 
where we have agreed I should have completed the project proposal for review.

Nearing the end of the week I have completed my project proposal and sent it to 
Stephen for him to review and discuss in our meeting on the 12th. I have fleshed out 
more of my thoughts towards how I should approach the project.

\section*{Week 3 Sem 1, 12th Oct - 18th Oct}
During this weeks meeting with Stephen he provided me with feedback on my project 
proposal and we discussed moving on to the literature review. His comments about the 
proposal were very useful and his advice for the literature review has helped me 
understand the structure that I should use when writing it. I will be focusing on 
bringing together and comparing the content of multiple papers with respect to specific 
parts of my project such as tree branch construction or leaf placement etc.

The end of the week was used to start the introduction to my literature review where 
I described the project and discussed the required knowledge for completion of the 
project.

\section*{Week 4 Sem 1, 19th Oct - 25th Oct}
This weeks meeting with Stephen was used to discuss the introduction I sent him for 
the literature review and how to improve it, along with how I should proceed with the 
rest of the document. I have a better understanding of how I will be approaching the 
rest of the literature review now. I have also downloaded several papers related to 
the project that I will use for my comparisons in the rest of the document, I hope to 
have it finished by my next meeting with Stephen on the 26th.

Progress on the document did not proceed as quickly as I would have hoped. After 
evaluating the papers I downloaded I found that a few were not strictly relevant to 
the project and so had to remove them, leaving me with less content to use. Other 
obstacles prevented much work being done on the literature review but it is not 
expected until the end of next week so I am still on target.

\section*{Week 5 Sem 1, 26th Oct - 1st Nov}
This weeks meeting with Stephen wasn't as productive as I would've hoped due to me not 
making as much progress on the literature review as I wanted to. We discussed some of 
the upcoming sections in the review giving me a better understanding of what I should 
consider for content.

The second half of this week proved quite productive where I managed to write a lot 
of content relating to branch structure, leaf placement and wind affect with respect 
to various papers. This process has greatly improved my understanding of many of the 
key areas of knowledge that will be required in the design and implementation stage.

\section*{Week 6 Sem 1, 2nd Nov - 8th Nov}
This weeks meeting was used to discuss the content I had written in the latter half of 
Week 5. Unfortunately the review was not finished for the end of Week 5 but good progress 
has been made and I will be able to finish it in the next couple of days.

I have now completed the literature review and sent off my final draft to Stephen, the 
last content added was a comparison of work related to detail management with tree models, 
such as different approaches to LOD drop off, and then a simple conclusion.

\section*{Week 7 Sem 1, 2nd Nov - 8th Nov}
During this weeks meeting Stephen and I discussed how I should proceed into the design and 
planning stage of the project and that including beginning to experiment with OpenGL 3D 
programming.

I have been making progress understanding some basics with OpenGL 3D programming 
gathering together the fundamentals for creating an application such as opening a window and 
setting clear colour. 

I will continue working to understand the general requirements for producing an OpenGL 
application during this week.

\section*{Week 8 Sem 1, 9th Nov - 15th Nov}
This week I have continued with more basic progress with the OpenGL application being able 
to load in objects and apply textures. I have also continued general research on various 
tree related 3D rendering methods to further my understanding. I have mostly been focussing 
on the work of Przemys\l{}aw Prusinkiewicz as his various work covers most facets of this subject.

\section*{Week 9 Sem 1, 16th Nov - 22nd Nov}
This week has been slower than previous as I have started working on producing generated shapes 
in my OpenGL application. I will need some sort of struct to generate lines for my branch 
structure that can be loaded in from certain points. 

I will continue to work on this and make as much progress as possible while beginning to write 
up the progress report for the project.

\section*{Week 10 Sem 1, 23rd Nov - 29th Nov}
This week I have been working on getting more general facets of the OpenGL application working 
such as a basic lighting setup and gaining a more thorough view of the OpenGL rendering 
pipeline which has helped me with understanding how the application should be written.

I have made decent progress on the progress report having added the introduction and the 
explanation for the problems of branch structure and leaf placement.

\section*{Week 11 Sem 1, 30th Nov - 6th Dec}
In this weeks meeting I received feedback from Stephen about the report and OpenGL program 
helping me understand how I should progress from here. 

I will be continuing to work on the progress report more having started on the design and 
planning stage of the report where I will discuss the method of L-systems. 

\section*{Week 12 Sem 1, 7th Dec - 13th Dec}
In this weeks meeting with Stephen we discussed the design and planning section I had started 
about L-systems. After some discussion we decided that I should rewrite this section with more 
detail for the L-system description.

I have done further research into L-systems while looking to write the design and planning stage 
and learned more about how I could possibly approach the implementation stage. I will likely 
change my design from using lines as a skeleton and then applying thickness afterwards. To 
producing an L-system that accounts for thickness of branches by using cylinders instead of lines.

\section*{Week 1 Sem 2, 1st Feb - 7th Feb}
Over the Christmas break I have been making some progress in the early stages of the implementation 
and testing how I should approach the tree construction problem. 

My plan is to use some kind of L-system for the branch construction but at the moment I am 
testing the best way to represent the branches. So far I have tested the use of gluCylinders 
which produce a decent result with simple input. However, I will most likely choose to make my 
own cylinder loading class so I have more control over the construction process. 

\section*{Week 2 Sem 2, 8th Feb - 14th Feb}
After discussing the choice of using gluCylinders or my own models with Stephen, we have come 
to the conclusion that making my own models would be more useful because I would be able to 
fully understand the generation process, and fix any problems that may appear while testing.

Having decided to use my own models rather than gluCylinders I spent some of this week putting 
together the relevant methods for constructing my own cylinder models such as the vertex 
buffer objects required. Time was spent making sure that I properly understood the underlying 
concepts required to produce this class. 

\section*{Week 3 Sem 2, 15th Feb - 21st Feb}
This week I finished up the basic cylinder construction methods but have decided to just use 
line construction for now. This is due to the cylinder geometry being very complicated to work 
with, and also because at the moment in the early stages of testing it will be easier to diagnose 
problems in construction when looking at simple lines. 

\section*{Week 4 Sem 2, 22nd Feb - 28th Feb}
To test the use of my line construction and to develop my understanding of L-systems, I have 
decided to reproduce some example L-systems found online. These L-systems are the binary tree 
and barnsley fern. The binary tree will be useful for testing rotations to make sure that I 
can manage the regular rotations required, and due to the binary tree being a very simple shape 
it will be easy to diagnose any problems. The barnsley fern will be a useful test to see if 
this line construction method will be appropriate for creating natural patterns. 

\section*{Week 5 Sem 2, 1st Mar - 7th Mar}
This week I finalised the basic binary tree and barnsley fern L-system generation which produce 
the expected results after some bug fixing. It seems that the most likely issue when translating 
these systems to OpenGL is how to properly manage rotations to produce the expected results. 

Having shared my results with Stephen I am now happy to move on to try and develop a more advanced 
L-system that will be able to generate 3D tree structures.

\section*{Week 6 Sem 2, 8th Mar - 14th Mar}
I have decided that for the advanced L-system I will adapt a parametric L-system from an essential 
paper I found during the literature review stage. This paper by Przemys\l{}aw Prusinkiewicz, 
Mark Hammel, Jim Hanan and Radom\'{i}r M\v{e}ch describes many essentials for advancing L-systems 
that will be useful in developing my understanding.

\section*{Week 7 Sem 2, 15th Mar - 21st Mar}
This week I spent some more time reading through the paper by Prusinkiewicz et al. and have learned 
more about L-systems as well as the example that I am going to adapt. This example from the paper 
is also well documented with some results from chosen inputs so this will assist in the testing 
process of my adaptation because I will be able to benchmark my results to that of the paper.

I have started to implement the logic required for this L-system and will try and produce it the 
same way I did with the binary tree and barnsley fern examples.

\section*{Week 8 Sem 2, 22nd Mar - 28th Mar}
This week I have still been working on adapting the advanced L-system. However, I have decided 
to stray from the classical approach of L-system string generation to instead use an object 
oriented approach. L-systems are commonly used to generate strings and then the resulting string 
is iterated through to perform the related task, in this case however the string is extremely 
complicated and would be very difficult to translate to generating the structure in OpenGL. 
The paper uses turtle graphics for it's representation, which removes the issues of cumulative 
rotations at following bifurcations due to the turtle having a ``heading'' that can be easily 
added to or subtracted from. Whereas, with OpenGL the rotations cannot be simply additive when 
traversing the string due to parent branch rotations possibly not coinciding directly with their 
child branches.

I have therefore decided to use an object called Apex (taken from how Prusinkiewicz et al. refer 
to the branch segments in the paper) to represent the branch segments with appropriate attributes 
that can be referenced such as rotations. Another benefit from using objects is that I can use 
parent child relationships between Apex objects to edit child attributes with relation to the 
parent Apex.

\section*{Week 9 Sem 2, 26th Apr - 2nd May}
Over the Easter break I have made good progress in creating Apex objects for tree construction.
I have been using examples from the Prusinkiewicz paper for reference and decided to try and 
create their version of the simple binary tree first. 

This proved relatively simple due to the even rotations and linear length degradation and I was 
able to get a result equivalent to that shown in the paper. Rotations have been the largest blocker 
when dealing with bugs due to visualising the problem being quite hard and issues not always being 
completely obvious.

For the latter half of the Easter break my focus then shifted to producing the other examples from 
the paper to prove that my structure generation was correct. It seems that trees only using alpha 
rotations work fine but once the phi rotations are introduced the structure completely breaks down.
I have been discussing various approaches with Stephen and at the moment progress has been slowed 
significantly. Small improvements are being made throughout the program however, while trying to 
diagnose the rotation issue I have identified various other issues that have now been fixed.

\section*{Week 10 Sem 2, 3rd May - 9th May}
This week I have continued to try and fix the phi rotation issue with various methods. As with 
last weeks summary there hasn't been an improvement in the phi rotations but other issues have 
been fixed and I have added some usability changes such as being able to choose the position of 
the tree in world space, and a better camera for viewing the trees that will be useful for 
demonstrating the project results.

The phi rotation issue has now been fixed after switching to a matrix rotation method the issue 
was revealed to be that the rotations meant to be about the y axis were not acting as expected. 
Now all examples from the paper have been successfully benchmarked and some acceptable tree 
branch structures can be produced using the system.

With recommendation from Stephen I have decided to end development and move onto writing the 
project report at this time. Unfortunately I was unable to add thickness and leaves to the 
trees, however I have laid the groundwork for branch thickness to be implemented and I know 
a likely method for how I would add leaves so I will discuss this in the evaluation section of 
the report.

\section*{Week 11 Sem 2, 10th May - 16th May}
This week I have been working on the report and am currently on schedule to finish the required 
content on time for Wednesday the 19th.

\section*{Week 12 Sem 2, 17th May - 19th May, Deadline}
The report and other supporting materials have been finalised and are ready to be submitted.

\begin{figure}[p]
        \begin{sideways}
        \newganttchartelement{voidbar}{
        voidbar/.style={
        draw=black,
        top color=black!25,
        bottom color=black!23
        }}
        \begin{ganttchart}[x unit=0.45cm, vgrid, title label font=\scriptsize,
        canvas/.style={draw=black, dotted},
        /pgfgantt/milestone left shift = 0,
        /pgfgantt/milestone right shift = 0
        ]{1}{34}
        \gantttitle{Project schedule shown for e-vision week numbers
         and semester week numbers}{34} \\
        \gantttitlelist{9,...,42}{1}\\
        \gantttitlelist{1,...,12}{1}
        \gantttitle{CB}{4}
        \gantttitle{AS}{2}
        \gantttitlelist{1,...,8}{1}
        \gantttitle{EB}{4}
        \gantttitlelist{9,...,12}{1}\\
        
        
        %the elements, bars and milestones, are identified as elem0, elem1, etc
        
        %elem1
        \ganttbar{Project proposal}{1}{2}     \\  %elem0  
        \ganttbar{Literature review}{3}{6}    \\  %elem1 
        \ganttbar{Design}{6}{11}              \\  %elem2
        
        %week 1 of semester 2 is the 17th week in schedule 
        \ganttbar{Coding}{12}{12}                 %elem3
        \ganttvoidbar{}{13}{16}                   %elem4
        \ganttbar{}{17}{20}                   \\  %elem5
        
        \ganttbar{Testing}{12}{12}                %elem6
        \ganttvoidbar{}{13}{16}                   %elem7
        \ganttbar{}{17}{26}                    \\ %elem8
        \ganttmilestone{Code delivery}{26}    \\ %elem9
        \ganttbar{Final report writing}{25}{30}        \\ %elem10
        \ganttmilestone{Portfolio submission}{33}  \\ %elem11
        \ganttbar{Inspection preparation}{31}{34}  %elem12
        
        
        \ganttlink{elem0}{elem2} \ganttlink{elem1}{elem2} \ganttlink{elem2}{elem3}
        \ganttlink[link mid=.25]{elem2}{elem6} \ganttlink{elem5}{elem6}
        \ganttlink{elem8}{elem9} \ganttlink{elem9}{elem10}
        \ganttlink{elem10}{elem11} \ganttlink{elem11}{elem12}
        \end{ganttchart}
        \end{sideways}
        \caption{Revised Gantt Chart from Project Proposal}
    \end{figure}

\end{document}