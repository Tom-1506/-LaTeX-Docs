\documentclass[review]{cmpreport}
\graphicspath{ {./images/} }

\title{GPU Accelerated Method for Constructing and Rendering Trees 
        \\ - \\ 
        Literature Review}
\author{Thomas Mcloughlin}
\date{12/10/2020}
\registration{100203952}
\ccode{CMP - 6013Y}
\supervisor{Dr. Stephen Laycock}

\begin{document}
\maketitle

\section{Introduction}
The rendering of trees has advanced greatly from the early days of real-time 
applications. The earliest method of adding trees to an environement would be 
to simply load in a flat image of a tree and then have the image rotate to face 
the view point of the camera. This was acceptable for single placed trees but 
for areas with many trees the likely approach would be to repeat the image 
around the central axis of the tree to give a sense of volume. \par
Developments in graphics hardware allowed for more detailed models to be used 
in real-time applications. This would've started with more simple manually 
modelled trees with actual branches and leaves applied using bill-boarded 
textures or sometimes individually for games with less demanding environments.
\par Presently, trees are added to real time environments as fully developed 
realistic looking models. Trees are constructed and rendered using software and 
then loaded into a scene, with the exception of some that may be modelled 
manually if a specific shape or look is needed for the tree. \par
This project is a similar system, aiming to be able to produce realistic 
and varied looking trees in a real-time application to help bring life to an 
environment. The following section will provide a brief description of the 
project, the areas of knowledge required to complete the project and a general 
roadmap for the rest of the document.

\subsection{Description}
Generating natural environments can be costly. Creating and rendering realistic 
models of trees can be challenging. This projects goal is to investigate 
approaches for creating and rendering trees to be used in a real-time graphics 
application. \par
A key reason for wanting to include trees in computer generated environments is 
that trees, and other foliage, are what give life to that environment, a forest 
is not a forest without the trees and having an easy method of including trees 
in a landscape will mean that making that landscape more realistic and engaging 
becomes easier.

\subsection{Knowledge}
The key areas of knowledge required to complete this project are as follows:
\begin{itemize}
      \item Branch Growth - The technical knowledge to produce natural looking 
            branches will be most important to the success of this project, 
            if the branches of the trees do not look naturally shaped then 
            the final tree will not look realistic. \par
            The method used for naturally formed branches will be decided from 
            researching multiple sources to find a suitable algorithm that 
            can be implemented as efficiently as possible.
      \item Leaf Placement - Similarly to the issue of branch growth, the 
            placement of leaves on the tree branches will also be important 
            with respect to making the rendered trees look realistic. \par 
            The method chosen will also similarly be researched from multiple 
            sources to find the most appropriate algorithm for efficient 
            implementation.
      \item Texturing - The textures applied to the trees after the geometry 
            is finalised will not be as essential as the previous points but 
            it will still play a key role in making the trees look realistic.
            \par A method will need to be used to correctly apply a bark 
            texture to the trunk and branches of the tree without jarring 
            edges being noticeable. The same attention will also need to be 
            given to the leaf texture and the decision of whether to include 
            some transparency in the leaves.
      \item Branch Pathing - This can be grouped with the branch growth section 
            as a possible extension. The idea that an obstacle could be used 
            to obstruct the growth of the tree's branches and the branch growth 
            algorithm would take this into account and render the tree to look 
            it has grown to avoid the obstacle. \par 
            This would help the rendered trees properly integrate with the 
            environment they're placed in and add a more realistic feel.
      \item Root System - A possible inclusion would be a ground level root 
            system for the trees, not a fully rendered underground system as 
            that would be pointless, but some of the roots of larger trees 
            will be visible around the base of the tree. \par 
            Some variation could be added to the trees by not having them 
            magically sprout from the ground and instead showing some of the 
            ground roots. A method would have to be chosen to model this, 
            possibly using the sam method for the branches but reversed.
\end{itemize}

\subsection{Roadmap}
The following sections of this document will include: comparisons of material 
related to the project to be used to inform the direction that will be taken in 
design and development, following the material comparisons - a discussion of 
the key issues and themes related to the project and how some of the discovered 
materials can be used to understand these facets, and finally an evaluation of 
the document containing a critical review of the content presented throughout.

\pagebreak
\section{Comparisons of Related Material}
This section will include a description of the methods used to find relevant 
materials to discuss around the project and a comparison of these materials 
where overlap is found.

\subsection{Search Method}
Relevant papers were found using google scholar and using search terms including 
"real-time", "rendering", "modeling" etc. effort had to be made to avoid papers 
referring to data structure related trees and focus was given to making sure that 
the chosen material was about real-time rendering in evironments and not just 
methods for generating premade models. \par
The different papers have varying focuses not all exactly about the rendering 
of trees, some were chosen to give some possibly needed background that will 
be relevant to the project less directly such as talking about fast rendering of 
large environments including trees. \par
Papers where not excluded based on date of release as the methods discussed are 
mostly algorithmic and not necessarily dampened by advances in technology, 
most relevant papers have been released withing the last 20 years however due 
to that being when hardware became powerful enough to begin this form of research 
in earnest. The papers chosen were also checked to have around 20 and above 
citations which should show that they have acceptable credibility.

\subsection{Comparison of Material}
In this section some of the key areas of required knowledge will be discussed by 
method with reference to various papers that have been found to contribute to each 
subject.

\subsubsection{Branch Structure}
The branching structure of rendered trees is the most commonly talked about subject 
across most papers relating to rendering trees. This is likely because the branch 
structure is most important with regards to making a tree look realistic, therefore 
it becomese the main focus of most relevant papers. \par
One of the earliest algorithmic methods of branch structure comes from 
\cite{honda1971description} where a ``mother'' branch splits into two ``daughter'' 
branches using given angles and having a determined length from a ratio with the 
``mother'' branch, this method can give various tree crown shapes providing a 
useful groundwork for tree construction. What's missing is the ability to calculate 
thickness of branches, or the possibility of producing branches that are not 
completely straight. \par
Another pioneering work for branching structures was the creation of L-systems, 
proposed by \cite{lindenmayer1968mathematical} and later discussed by 
\cite{prusinkiewicz1996systems}, which is a more general method of producing 
fractal data but can be applied to graphical drawing techniques to produce 
relatively realistic looking branching structures. It works by using a dictionary 
of variables, represented as characters, that are assigned to certain drawing techniques, 
commonly examples using turtle graphics, and then from a given axiom the variables 
are built upon using certain rules. Once a chosen number of recursions is achieved 
the resulting string is then used to direct the drawing. With care taken to what 
rules are used this method can be successful in producing many different kinds of 
branching shapes. The use of L-systems also allows for the creation of bending 
branches unlike the method presented by Honda, however the methods presented by 
honda could be included in the ruleset of an L-system to produce trees of a simiar 
structure. Both Lindenmayer and Honda's papers are both included in the Journal of 
Theoretical Biology. \par
\cite{weber1995rendering} and \cite{runions2007colonization} both make reference to 
Lindenmayer and Honda's work with their presentation for branching structures. 
Weber and Penn present the construction of a tree using truncated cones for ``mother'' 
branches and cones for terminating branches, by using cones they avoid the drawback 
of Lindenmayer and Honda's original methods that only produced a skeletal structure. 
They use a similar method for splitting ``mother'' branches into dichotomous 
``daughter'' branches but presents a much more detailed model allowing for more 
control of the various facets of the tree. \par
Prusinkiewicz et al. suggest the use of ``attraction points'' which are used to 
guide the growth of a tree skeleton into a desired shape where each point has an 
area of attraction that applies some value to any branches within it's radius, this 
causes the branches to grow towards the points until all points are reached or a 
chosen number of iterations is completed. The method referenced by Prusinkiewicz 
et al. for giving the tree skeleton thickness is from \cite{bloomenthal1985modeling} 
where the branches of the tree skeleton are treated as the centre point of many 
generalized cyclinders with varying radii, this requires the tree skeleton construction 
method to include the data for normals along sections of the branches so that the 
cyclinders can be constructed properly. \par
This method of generalized cyclinders could be applied to an implementation of Honda's 
method or a chosen L-system to give those trees thickness, avoiding the use of the 
cone method presented by Weber and Penn that may look too artificial. Bloomenthal 
also suggests the use of splines instead of lines, for the branches, to make the trees 
look more natural. \par
A common part of the methods posed by both \cite{weber1995rendering} and 
\cite{runions2007colonization} that hasn't been mentioned is the use of a terminal 
enveloping of the tree skeleton to prevent the branches from growing out too far or to 
restrict the tree crown into a preferred shape, Prusinkiewicz et al. refer to this as 
just an envelope whereas Weber and Penn call this overall method ``Pruning''. \par
A later paper by \cite{weber2008simulation} presents another similar construction method 
to those mentioned above but with more attention paid to the possible simulation of the 
rendered trees. He puts forward a method for branches avoiding obstacles using collisions 
with environment objects, this is done by pushing any branches that collide with the given 
object away from its bounding area by changing their branching angles.

\subsubsection{Leaf Placement}


\clearpage
\bibliography{bibfile}
\end{document}