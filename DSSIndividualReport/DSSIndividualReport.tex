\documentclass{ueacmpstyle}

\RequirePackage{natbib}
\usepackage{graphicx,caption}
\usepackage{algorithm}
\usepackage{algpseudocode}
\usepackage{appendix}
\usepackage[hyphens]{url}

\begin{document}
	\title{Developing Secure Systems Individual Report}
	\author{
		100203952 -- Thomas Mcloughlin\\
		CMP-6045B
	}
	\maketitle

    \section{Introduction}\label{sec:Intro}
    This report will present research of the top cyber threats and vulnerabilities of modern 
    systems with the aim of identifying methods to mitigate them.

    \section{Part 1}\label{sec:Pt1}
        % \cite{bibItem} << Example cite command (Look at overleaf tutorials) 
        % \ref{app:A} << This example references appendix A (Look at overleaf tutorials) 
        
        \subsection{Account Enumeration}\label{sub:AccEnum}
        Account enumeration is the manipulation of a service's login function to determine 
        the existence of a user. Attackers would determine this through two actions, first 
        by inputting a username and password into the login and if a message is received 
        stating that the password is wrong but not the username, then the attacker now 
        knows that the username exists. The way to mitigate this manipulation is to return 
        a generic message for a login failure not specifying whether the username or 
        password is wrong, however this mitigation can be nullified using the second method 
        where the attacker will try to log in and then compare the time taken to resolve the 
        failed login. If the response was quick then the username doesn't exist, if the 
        reponse took slightly longer then the service recognised the username and took longer 
        to try and match the password. This manipulation can also be mitigated by applying a 
        delay to the reponse when the username is wrong so that the attacker could not 
        tell the difference between the two responses.
        The mitigation methods described above fit into a secure by design approach because 
        they are concerned with the back end implementation details of the system and are 
        obscured well from any attackers. These techniques have been represented as 
        pseudocode in section \ref{sec:account-enumeration} of Appendix A.

        Threat actors likely to use account enumeration are attackers of any kind that could 
        range from casual programmers to criminal hackers trying to access various services. 
        A likely attack vector for using attack enumeration would be if the attacker had 
        gained access to a list of passwords for a service and was trying to find users to 
        match to so that they could break into the system.

        The interaction between the end users and the service will not be greatly affected 
        by the implementation of these mitigations with respect to the improved security 
        they provide, these mitigations do not affect the process of a sucessful login. 
        The only problem that can arise with usability will be that if a user forgets their 
        password or isn't sure what username they used for the service, the generic message 
        not specifying which is wrong can be frustrating and make logging in more difficult. 

        
    \section{Part 2}\label{sec:Pt2}

    \section{Conclusion}\label{sec:Con}
    
    \bibliographystyle{apalike}

    \nocite{*}
	\bibliography{bibfile.bib}
	
	\newpage
	
	\appendix
	    % Using appendices in this format means you can just use \section{}
        \section{Appendix A}\label{app:A}   % Adding the label will allow you to reference this section in your work.
            \newpage

            \begin{figure}[ht]
                \subsection{Account Enumeration Mitigation}
                \label{sec:account-enumeration}
                \centering
                \begin{algorithm}[H]
                \caption{login(\emph{username}, \emph{password}) {\bf return} \emph{response}}
                      \begin{algorithmic}[1]
                          \Require $username$, the username for login
                          \Require $password$, the password for login
                          \Require $users$, the set of users and passwords that the system will 
                                    compare the login against
                          \Ensure \emph{response}, either a sucessful login or a response message
                          \State{$Failure \leftarrow$ "The username/password is incorrect"}
                          \ForAll{$user$ in $users$}
                            \If{$username$ = $user.username$}
                              \If{$password$ = $user.password$}
                                \State{$response \leftarrow Sucess$}
                                \Comment{Username and password correct}
                              \Else
                                \State{$response \leftarrow Failure$}  
                                \Comment{Username correct, password incorrect}
                              \EndIf
                            \Else
                            \State{$delay$} 
                            \Comment{Wait however long the check for the password would take}
                            \State{$response \leftarrow Failure$}
                            \Comment{Username and password incorrect}  
                            \EndIf
                          \EndFor
                          \State \Return $response$
                      \end{algorithmic}
                \end{algorithm}
                \caption{Example of a login with account enumeration mitigation included}
                \label{fig:account-enumeration}
            \end{figure}

        \section{Appendix B}\label{app:B}

\end{document}
