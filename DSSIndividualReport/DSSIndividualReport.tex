\documentclass{ueacmpstyle}

\RequirePackage{natbib}
\usepackage{graphicx,caption}
\usepackage{algorithm}
\usepackage{algpseudocode}
\usepackage{appendix}
\usepackage[hyphens]{url}
\graphicspath{ {./images/} }

\begin{document}
	\title{Developing Secure Systems Individual Report}
	\author{
		100203952 -- Thomas Mcloughlin\\
		CMP-6045B
	}
	\maketitle

    \section{Introduction}\label{sec:Intro}
    This report will present research of the top cyber threats and vulnerabilities of modern 
    systems with the aim of identifying methods to mitigate them.

    \section{Part 1}\label{sec:Pt1}
      
      \subsection{Account Enumeration}\label{sub:AccEnum}
      Account enumeration is the manipulation of a service's login function to determine 
      the existence of a user. Attackers would determine this through two actions, first 
      by inputting a username and password into the login and if a message is received 
      stating that the password is wrong but not the username, then the attacker now 
      knows that the username exists. The way to mitigate this manipulation is to return 
      a generic message for a login failure not specifying whether the username or 
      password is wrong, however this mitigation can be nullified using the second method 
      where the attacker will try to log in and then compare the time taken to resolve the 
      failed login. If the response was quick then the username doesn't exist, if the 
      reponse took slightly longer then the service recognised the username and took longer 
      to try and match the password. This manipulation can also be mitigated by applying a 
      delay to the reponse when the username is wrong so that the attacker could not 
      tell the difference between the two responses.
      The mitigation methods described above fit into a secure by design approach because 
      they are concerned with the back end implementation details of the system and are 
      obscured well from any attackers. These techniques have been represented as 
      pseudocode in section \ref{sec:account-enumeration} of Appendix A.

      Threat actors likely to use account enumeration are attackers of any kind that could 
      range from casual programmers to criminal hackers trying to access various services. 
      A likely attack vector for using attack enumeration would be if the attacker had 
      gained access to a list of passwords for a service and was trying to find users to 
      match to so that they could break into the system.

      The interaction between the end users and the service will not be greatly affected 
      by the implementation of these mitigations with respect to the improved security 
      they provide, these mitigations do not affect the process of a sucessful login. 
      The only problem that can arise with usability will be that if a user forgets their 
      password or isn't sure what username they used for the service, the generic message 
      not specifying which is wrong can be frustrating and make logging in more difficult. 

      \subsection{Session Hijacking}\label{sub:SessHijk}
      Session hijacking is the utilisation of a lack of security given to website sessions 
      where a user that has logged in creates a session with the web server so that they 
      can make requests to the server without having to send log in details for every 
      request. These sessions have an attached ID so that the web server knows which user 
      it is communicating with, session hijacking refers to the multiple methods used to 
      gain access to a session, usually by accessing the ID. 

      When attempting to gain access using the ID, one possibility could be that the site 
      uses existent session ID's rather than generating a new ID for every session. This 
      opens up the session for attack from session fixation where the attacker uses a 
      known ID in a phishing email link to have the user login and authenticate themselves 
      then the attacker can hijack the session using the session ID \citep{OWASPSessionFixation}.

      There are multiple methods used to acquire session ID's to use for session fixation. 
      Session sniffing, the use of packet sniffing software to intercept session packets 
      and acquire the session ID attached to it, this can be done manually by the attacker 
      or the attacker may use malware to automate the process. The attacker may also brute 
      force the ID's by going through all possible permutations of the ID.

      The threat actors likely to use session hijackers are the same as stated above 
      in \ref{sub:AccEnum}. The risk posed by these attacks are: the attackers would 
      be able to gain access to a user authenticated login and perform any actions that the 
      user would be able to within that session such as a money transfer, the attacker would 
      also have access to any personal information that the session allows the user to view 
      which may lead to ID theft, the attacker may encrypt valuable/vital data for ransom 
      which could include intellectual property.

      The main methods to mitigate session hijacking attacks include: making session ID's 
      long and complex to avoid brute force access, make the site use a new session for 
      each time a user logs in and give each new session a unique ID to stop access if a 
      previously used ID has been compromised, ensure that a session is closed once a user 
      logs out or if the session is not being used and times out and finally, all session 
      data should be encrypted to prevent sniffing and malware attacks from accessing the 
      session ID's. Another method of mitigation extraneous to any technical measures 
      would include the training of end users to spot and avoid phishing emails.

      Generating a unique ID with enough complexity to avoid ID guessing is very important. 
      Therefore an example ID generation algorithm has been included as an acceptable 
      method shown in section \ref{sec:session-id-gen}. This algorithm was created using 
      recommended practices from \cite{OWASPSessionManagement} such as ensuring the entropy 
      number used is 64 bits to give an acceptable level of complexity.

      The addition of any of the mentioned mitigations would have very little affect on 
      the end users as the mitigations proposed are mainly secure by design techniques 
      that are more concerned with backend interaction. Usability will not be sacrificed 
      to a noticeable degree, the only affect on the user would be the requirement to 
      always log back into the system once they leave as the session they previously used 
      would have been closed.

      \subsection{SQL Injection}\label{sub:SqlInjection}
      SQL Injection is the manipulation of sql queries to interact with a database in 
      a way not intended by design, allowing the attacker to view, modify and delete 
      data from the database. These malicious queries are manipulated using input 
      fields such as the username and password inputs on a login page. If the login 
      input fields take any value inputted and inserts that input into an sql query 
      meant to retrieve user data, an example of which can be seen in section 
      \ref{sec:sql-query} of appedix A. If the attacker inputs part of a valid query 
      that will always evaluate to true, such as ' OR 1=1;--, then the example query 
      will return all user data from the table. Depending on how much sensitive data 
      is stored in the user table this could be a very dangerous breach, for example 
      if passwords were also stored in the user table.

      There are multiple methods of mitigation for sql injection that will be discussed 
      below. Firstly is the use of prepared statements where instead of inserting input 
      data into sql queries dynamically, the programmer defines all sql queries before-
      hand with inputs being parameterised. Any input then passed through to that query 
      avoids executing an unintended query and will instead take the entire input as a 
      string, for the example mentioned above using ' OR 1=1;-- the query would merely 
      search for a username matching the string "' OR 1=1;--" \citep{OWASPSqlInjectionPrevention}. 

      Secondly the use of stored procedures which act in a similar way to prepared 
      statements to avoid dynamic sql generation and instead parameterise and validate 
      input. The difference between the two is that stored procedures are sql queries 
      stored in the database that are then retrieved by the application when needed. 
      If implemented safely, avoiding the use of dynamic sql generation, then stored 
      procedures can provide nearly the same protection as prepared statements. 
      However a possible vulnerability of stored procedures comapared to prepared 
      statements would be in the case of a database requiring certain access levels 
      such as read and write permissions to prevent certain use of the database. In 
      the case of stored procedures, access to execute queries on the database is 
      required meaning that if an attacker were to gain access to the database then 
      they would have full execute privileges \citep{OWASPSqlInjectionPrevention}.

      Another method would be the implementation of whitelist validation for certain 
      parts of sql queries that cannot use bind variables as placeholders such as the 
      names of tables or columns. In this case if a user inputted parameter is used to 
      alter a queries target table or column then whitelist validation can be used to 
      avoid the risks of sql injection. By mapping input values to a preset whitelist 
      of possible choices for the target, the executed query avoids using the actual 
      input from the user, an example is described in section \ref{sec:whitelist-valid} 
      of appendix A.

      The mitigation methods discussed above would all be necessary inclusions in a 
      secure by design approach as it would take considerable effort to convert a system 
      to use these methods without a near full re-write of the related code. When trying 
      to combat sql injection attacks, mitigation should be considered from step 1 and not 
      be a reactionary change to a system.

      The threat actors likely to use sql injection are again the same as those stated 
      in \ref{sub:AccEnum}. The main risk of sql injection is that an attacker could be 
      able to execute any query they want on a database, in the case of a catastrophic 
      breach, the attacker could delete tables containing important data.

      End-users are unlikely to experience any adverse affects of the mitigation methods 
      as all methods used to prevent sql injection are backend related, these mitigations 
      would not affect usability of the system a noticeable amount to the common user. 
      The only affect on the system will be the minor performance concerns.

      \subsection{Cross-site Scripting}\label{sub:XSS}
      Cross-site scripting refers to the various methods used to execute malicious 
      javascript to access or steal information from a victim. There are 3 types of XSS, 
      the first being "Stored". Stored cross-site scripting is the most damaging type of 
      XSS where an attacker uses a website form, commonly on a website that allows public 
      posts such as a social media site, to insert a malicious string into the websites 
      database. This malicious string is now stored in the site, hence the name, and when 
      a victim requests the compromised post, the website includes the malicious string in 
      the response and sends it to the victim. The victims browser then executes the script 
      inside the response. The script can be used for various functions but is commonly used 
      to steal cookie data so that the attacker can then gain access to the site using the 
      victims access, this is session hijacking as explained in section \ref{sub:SessHijk}. 
      This form of XSS is particularly dangerous as one insertion of malicious script can 
      affect multiple victims due to the script having been stored in the website database 
      and multiple victims may request that data.

      The second type of XSS is reflected cross-site scripting which differs from stored 
      by having the malicious script be sent to the victim first in the form of a link to 
      a site. The attacker tricks the victim into clicking the link which sends the 
      malicious script to the vulnerable site, the site takes the data from the request and 
      adds the malicious script to the webpage where it then executes the script which could 
      be used for the same uses as described above, commonly session hijacking.

      The third type of XSS is done through manipulation of a websites document object model 
      (DOM). The attacker constructs a URL containing the malicious script and sends it to 
      the victim, the victim is tricked by the attacker into requesting the URL from a site 
      that is vulnerable to attack. The website receives the request, but instead of 
      responding to the victim with the malicious script, the DOM of the website is manipulated 
      to insert the malicious script into the website and then the victims browser executes it. 
      This sends the victims cookies and sensitive data to the attacker allowing them to hijack 
      their session or use the sensitive data for other means such as identity theft.

      The first important method of mitigation for XSS relating to stored and reflected XSS is 
      encoding website element content also known as escaping. If the attacker has inserted 
      some malicious script into a post on a social media site for example. The site should 
      check the string used for the post and character by character, encode any possibly 
      malicious characters into a different format to avoid the script being seen and 
      executed as javascript. For example the "$<script>$" section of the malicious string 
      may be encoded in ASCII, so the '$<$' and '$>$' characters will be converted to \&\#60 
      and \&\#62 respectively. To avoid loss of content in the actual post, the html element 
      used to display the content should then be set to use ASCII encoding . This will result 
      in the script being shown as part of the post and the user may read it, but it will not 
      be recognised as script and executed \citep{IBMProtectFromXSS}. Escaping for different 
      expected characters is used when needed in certain contexts, the example described 
      above is an HTML escape context, but it important to do the same for Javascript and 
      others when needed. An example of how this encoding could be designed is shown in 
      section \ref{fig:xss-char-esc} of appendix A. 

      Another method of mitigation for XSS is validation of user input. Any data received 
      that originates from outside the system should be untrusted until validated. This 
      validation could be done using a whitelist of known acceptable inputs depending on 
      the context.
      Along with validating input any untrusted input should also be sanitised to remove 
      unwanted data, depending on the context, including html tags and unsafe characters. 

      Mitigation for DOM cross-site scripting is more complex than for stored and reflected 
      XSS. When inserting untrusted data into sections of the DOM you should run HTML and 
      Javascript escaping before-hand as described above. However if the untrusted data is 
      inserted into the Javascript section of the DOM then encoding it will not prevent it 
      from being executed. A fundamental method of avoiding this issue is the use of a safe 
      Javascript assignment property $textContent$ and other safe content inserts, an example 
      of which can be seen in section \ref{sec:textContent} of appendix A. It is important 
      to avoid the use of innerHTML as an output method and instead use innerText of 
      textContent as explained above. This help prevent DOM based XSS vulnerabilities 
      \citep{OWASPDOMXSSPrevention}. 

      The mitigation techniques explained above are all important to a secure by design 
      approach, handling of untrusted data that is received by a site is incredibly important 
      to prevent manipulation of users access and personal information.

      The likely threat actors to use cross-site scripting attacks are once again mostly the 
      same as those stated in \ref{sub:AccEnum}, however cross-site scripting is likely to be 
      used more by those with an advanced understanding of hacking and website vulnerabilities. 
      The common attack vectors for cross-site scripting require the tricking of the user to 
      accessing a site through an attacker provided link, phishing emails being the most common 
      delivery system. The consequences of a successful attack can range from minor to 
      catastrophic, from simple redirection of the browser to cookie theft used for session 
      hijacking, keylogging to steal personal information and credentials, fake login forms also 
      for stealing credentials and using XSS to steal CSRF tokens that will be explained in 
      the next section.

      The end users can be affected by some of the mitigation techniques mentioned if not 
      properly implemented. Encoding of scripts being sent to a site can prevent certain 
      actions the site could make that may help usability. However the security benefits 
      of avoiding XSS attacks are numerous.

      \subsection{Cross-site Request Forgery}\label{sub:CSRF}
      Cross-site request forgery (CSRF) is a method used by attackers to execute requests on 
      a trusted website by using an already open session of a victim. If the victim has a 
      session open with a trusted site, then visits a site controlled by the attacker and 
      activates a link, by clicking a button for example, the link can make a request to 
      the trusted site using the open session that the user has causing some action to be 
      carried out by the site that the victim may not realise. An example would be if a 
      victim is logged into their bank account on the banks website, then accesses an attacker 
      owned site through a phishing email, the attacker could create a link that makes a request
      to the bank website to make a money transfer to the attackers account and because there 
      is an open trusted session with the victims account, the transfer would go through.
      
      One mitigation technique for CSRF is the use of tokens for site requests, this attaches 
      a server side generated token to either the user session or a separate token for each 
      field input, the latter being more secure, so that when the server receives a request 
      it can check for the existence of a token in the request or check that the token matches 
      to validate whether the request is legitimate or not. If the request is found to be 
      illegitimate then the current user session should be closed and logged as a possible 
      CSRF attack. CSRF tokens should be completely unique to avoid the capture and reuse of 
      previous tokens, they should be secret and they should be unpredictable, generated with 
      a suitable prng to prevent sequential brute forcing of token values. The tokens should 
      not be transfered via cookies and should instead use hidden fields and headers, the 
      tokens could also be encrypted to add another layer of security \citep{OWASPCSRFPrevention}.

      Another mitigation method is the use of captchas which help prevent spamming requests 
      to a server aswell as CSRF. By requesting an input that is random and unknown to the 
      attacker you place a blackage between their malicious request and execution. 
      Alongside captchas can be the use of re-authentication for certain actions, for example 
      requiring the user to input their password again when trying to make a bank transfer. 
      This can be taken another step further by requiring 2-factor authentication to avoid 
      the attacker being able to re-authenticate the request if the victims password has been 
      compromised.

      Also commonly used is the double submit cookie technique which involves sending a random 
      value in both a cookie and as a request parameter, the server then verifies if the 
      cookie value and the request value match. When a user visits, the site should generate 
      am cryptographically strong pseudorandom value as a hidden form value and set that as 
      a cookie in the users browser, separate from the session ID. The site then requires that 
      every request must include this random value as a hidden value. If both values match at 
      server side then the server accepts the request, if not then the request is rejected 
      \citep{OWASPCSRFPrevention}.

      These mitigation techniques are all good examples of secure by design, ensuring that 
      requests made to the web server are via the user. The end user will only have a slight 
      change to usability when interacting with sites using these techniques, for example the 
      use of tokens can cause the "back" function in a browser to not work as intended due to 
      the previously used token having changed or no longer being valid \citep{OWASPCSRFPrevention}.
      
      The most likely attack vector for CSRF is through phishing emails tricking victims into 
      accessing the attacker owned site. The risks involved could include: financial loss, 
      stolen data, changing of credentials, ID fraud and much more. 
        
    \section{Part 2}\label{sec:Pt2}
    In this section a comparison and evaluation between three authentication methods will 
    be carried out to determine which would be most suitable to implement in the group web 
    application. The methods chosen are the standard username/password, biometric 
    fingerprint scanner and graphical passwords.
    
    First will be an exaplanation of each method. The standard username/password method is 
    the most ubiquitous authentication method across all types of software. It uses a simple 
    pairing of a unique and semi-public username (sometimes an email address is used) with a 
    secret password. The semi-public nature of the username meaning that it is left 
    uncensored when inputted and is also used to refer to the user in communication and, in 
    the case of an email address, used to communicate with the user directly. The username 
    should be treated as secure information and withheld from unauthorised viewing where 
    possible, but it is not considered as important and keeping the users password safe. 
    \\
    There are many precautions put in place to keep a users password secret to avoid having 
    an attacker gain access to the users account. The most simple of which is to censor the 
    password when inputting to avoid shoulder surfing and having someone read your password 
    as you type it in. Some methods of secrecy are placed in the responsibility of the user 
    such as the use of special characters and character length requirements to make passwords 
    harder to guess or brute force, these requirements are usually built into the login 
    system to force users to create stronger passwords. 
    \\
    For back end related security there's the discussion of storage methods when handling 
    usernames and passwords. Once stored in a properly secured database you would decide 
    what hashing algorithm to use for the password and possibly the username aswell. 
    Algorithms such as MD5 are not suitable as they are reversible, this site \citep{NISTHashFunctions} 
    by the U.S National Institute of Standards and Technology recommends that SHA-256 
    should be used at minimum to provide a secure and unique hash. 

    Biometric fingerprint scanners work by taking an image/scan of the users fingerprint 
    and storing it, then when the user tries to login another scan is done and compared to 
    the saved scan to determine whether the user is authorised. Security needs to be 
    considered with how the users fingerprint is stored so that they cannot be stolen and 
    used for identity theft, an issue with this being the difficulty to securely store an 
    image for this purpose. The fingerprint scanner will not always receive exactly the 
    same image of the fingerprint, software has to find the similarities between the 
    inputted fingerprint and the stored image to decide whether the user is authorised. 
    This means that you cannot hash the stored fingerprint for comparison because the 
    inputted fingerprint will be different and produce a completely different hash that 
    cannot be compared. The hardware used for scanning also needs to be able to take 
    high enough fidelity images to recognise the unique differences between each fingerprint 
    to avoid false positives.

    Graphical passwords work by having the user choose a varying number of points on an image, 
    this image could be chosen by them, which are then stored. When the user wants to login 
    they must click the same points on the image that they clicked before, allowing for some 
    small margin of error because it will be too difficult to click the exact same pixel. 
    A similar issue to that explained above for storing the fingerprint arises with a 
    graphical password system. You must store the coordinates of the points inputted by the 
    user but you cannot hash them as you will need to compare the new points that the user 
    will click when trying to login.

    The reason that fingerprint scanning and graphical passwords were chosen for comparison 
    to the standard username/password method are because they show two directions that 
    technology has taken when trying to improve on the standard method. Both have tried to 
    take what works about username/password and improve where they can, one being successful 
    and widespread (fingerprint scanning) and the other not really moving into mainstream use 
    (graphical passwords).

    Some of the main reasons for trying to improve on the username/password standard is for 
    improving usability, using a fingerprint scanner is now extremely quick with modern scanners 
    and graphical passwords make it easier for users to remember there login as the human 
    brain is much better at remembering images than it is at remembering words \citep{Grady2703}.
    The usability of the standard username/password comes from it being so ubiquitous and 
    familiar to the common user, every computer will have at least a keyboard allowing 
    the input of a username and password. Username and password is slow however compared to 
    fingerprint scanning and a graphical password. It also requires remembering a complex 
    password compared to the easier to remember graphical password and not having to recall 
    any information for a fingerprint scanner. 
    \\
    The security trade-offs that go along with these usability points are as follows. For 
    a username/password approach the strong and ubiquitous security comes with the drawback 
    that users are likely to forget their complex password. When given the requirement to 
    have a complex and difficult to remember password, users are going to write down 
    passwords to help them login, therefore negating a main advantage of the security.    
    \\
    While the fingerprint scanner is fast and easy to use, it can be easily obscured if 
    there is dirt on the fingerprint or if there is damage to the fingerprint the user may 
    be unable to login at all (this should be and is avoided using multiple fingerprints), 
    the insecure nature of storing the fingerprint image, mentioned above, is also an 
    important factor. 
    For the benefit of security the fingerprint for logging in is usually stored on the device 
    rather than on a database server to prevent an attacker from stealing biometric data if 
    the database is compromised, however this comes with the drawback that you must have a 
    different stored fingerprint on each device. 
    \\ 
    With a graphical password, The issue of complexity when choosing an image is important, 
    with a very complex image, users are likely to forget where they clicked before and users 
    are also likely to pick the same points if they are using the same image. Even if an 
    attacker had never seen the image the user chose before gaining access to it, they would 
    likely be able to brute force guess some of the chosen points.

    For the group web application assignment I believe that using the standard username/
    password method will be most suitable. It provides a high level of security when using 
    complex requirements, doesn't require special hardware like a fingerprint scanner, allows 
    storing passwords securely using hashes unlike a fingerprint or graphical password image 
    and will be familiar to all users. 

    \section{Conclusion}\label{sec:Con}
    In this document I have discussed the common vulnerabilities of web applications in Part 1 
    with methods of mitigation for each, the threat actors likely to use these vulnerabilities, the attack 
    vectors that they would try to exploit, the risk implications of each vulnerability if 
    they were successfully attacked, how the mitigation methods relate to secure by design 
    techniques and the effect on end-users with usability vs security concerns.
    \\
    In Part 2 I discussed username/password, biometric fingerprint scanning and graphical 
    passwords as various authorisation techniques, how each of them work, their effectiveness 
    , the trade-offs with usability and security and which I recommend for implementation in 
    the group web application and why.
    
    \bibliographystyle{apalike}

  \newpage

	\bibliography{bibfile.bib}
	
	\newpage
	
	\appendix
	    % Using appendices in this format means you can just use \section{}
        \section{Appendix A}\label{app:A}   % Adding the label will allow you to reference this section in your work.

            \begin{figure}[ht]
              \subsection{Account Enumeration Mitigation}
              \label{sec:account-enumeration}
              \centering
              \begin{algorithm}[H]
              \caption{login(\emph{username}, \emph{password}) {\bf return} \emph{response}}
                \begin{algorithmic}[1]
                  \Require $username$, the username for login
                  \Require $password$, the password for login
                  \Require $users$, the set of users and passwords that the system will 
                            compare the login against
                  \Ensure \emph{response}, either a sucessful login or a response message
                  \State{$Failure \leftarrow$ "The username/password is incorrect"}
                  \ForAll{$user$ in $users$}
                    \If{$username$ = $user.username$}
                      \If{$password$ = $user.password$}
                        \State{$response \leftarrow Sucess$}
                        \Comment{Username and password correct}
                      \Else
                        \State{$response \leftarrow Failure$}  
                        \Comment{Username correct, password incorrect}
                      \EndIf
                    \Else
                    \State{$delay$} 
                    \Comment{Wait however long the check for the password would take}
                    \State{$response \leftarrow Failure$}
                    \Comment{Username and password incorrect}  
                    \EndIf
                  \EndFor
                  \State \Return $response$
                \end{algorithmic}
              \end{algorithm}
              \caption{Example of a login with account enumeration mitigation included}
              \label{fig:account-enumeration}
          \end{figure}

          \begin{figure}[ht]
            \subsection{Session ID Generation}
            \label{sec:session-id-gen}
            \centering
            \begin{algorithm}[H]
            \caption{generateUniqueSessionId() {\bf return} \emph{sessionId}}
              \begin{algorithmic}[1]
                \Require $IdList$, the set of existing session Id's
                \Require $prng$, pseudo random number generator
                \Ensure \emph{sessionId}, a unique session Id
                \State $entropy \leftarrow prng(64)$
                \Comment{Pseudo random number generation of length 64 bits}
                \State $Id \leftarrow generateId()$
                \Comment{Generate complex Id}
                \State $sessionId \leftarrow concatenate(Id, entropy)$
                \If{$sessionId \in IdList$}
                  \State generateUniqueSessionId()
                  \Comment{Session Id isn't unique, generate a new Id}
                \EndIf
                \State \Return $sessionId$
              \end{algorithmic}
            \end{algorithm}
            \caption{Example of a random session Id generator using recommended practices 
                      from \cite{OWASPSessionManagement}}
            \label{fig:session-id-gen}
          \end{figure}

          \begin{figure}
            \subsection{Basic User Retrieval SQL Query}
            \label{sec:sql-query}
            \includegraphics{basicSqlQuery}
            \caption{Example of a query that could be used to retrieve user data}
          \end{figure}

          \begin{figure}[ht]
            \subsection{Whitelist Validation Check}
            \label{sec:whitelist-valid}
            \centering
            \begin{algorithm}[H]
            \caption{getTableName(\emph{userInputValue}) {\bf return} \emph{outputTableName}}
              \begin{algorithmic}[1]
                \Require $userInputValue$, the value for the table name inputted by the user
                \Require $tableNameList$, list of valid table names
                \Ensure \emph{outputTableName}, the whitelisted chosen table name
                \ForAll{$name$ in $tableNameList$}
                  \If{$userInputValue$ = $name$}
                    \State $outputTableName$ = $name$
                  \EndIf
                \EndFor
                \State \Return $outputTableName$
                \Comment{\parbox[t]{.5\linewidth}{
                  if $outputTableName$ returns empty then we know that the user input was invalid
                }}
              \end{algorithmic}
            \end{algorithm}
            \caption{Example of table name validation, an adapted example from 
                     \cite{OWASPSqlInjectionPrevention}}
            \label{fig:whitelist-valid}
          \end{figure}

          \begin{figure}[ht]
            \subsection{External Script Character Escaping}
            \label{sec:xss-char-esc}
            \centering
            \begin{algorithm}[H]
            \caption{stringCharEscape(\emph{inputString}) {\bf return} \emph{encodedString}}
              \begin{algorithmic}[1]
                \Require $inputString$, the inputted string (potentially malicious)
                \Require $charEncodingList$, list of characters encoded values
                \Ensure \emph{encodedString}, the final encoded string
                \ForAll{$char$ in $inputString$}
                  \If{$char.encode \in charEncodingList$}
                    \State $encodedString \leftarrow concatenate(encodedString, char.encode)$
                  \Else
                    \State $encodedString \leftarrow concatenate(encodedString, char)$
                  \EndIf
                \EndFor
                \State \Return $encodedString$
              \end{algorithmic}
            \end{algorithm}
            \caption{Example of character escaping to encode a possibly malicious string}
            \label{fig:xss-char-esc}
          \end{figure}

          \begin{figure}
            \subsection{Use of textContent for Inserting Untrusted Data}
            \label{sec:textContent}
            \includegraphics{textContent}
            \caption{Example of using textContent to insert untrusted data safely into the DOM
                     \citep{OWASPDOMXSSPrevention}}
          \end{figure}

\end{document}
