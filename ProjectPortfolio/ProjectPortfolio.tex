\documentclass[final]{cmpreport}
\graphicspath{ {./images/} }

\title{GPU Accelerated Method for Constructing and Rendering Trees}
\author{Thomas Mcloughlin}
\date{21/3/2021}
\registration{100203952}
\ccode{CMP - 6013Y}
\supervisor{Dr. Stephen Laycock}

\summary
{
This project aims to convert and extend the Lindenmayer-system based tree construction 
method presented by \citep{prusinkiewicz1996systems} to be used as an independent 
OpenGL module. 
The module should allow the addition of trees to a real-time environment with minimal 
user interaction, avoiding the difficulties and expenses of manually producing tree 
models.
}

\acknowledgements
{
I'd like to thank my supervisor, Stephen Laycock for his brilliant help and support 
throughout this project.
Thank you to George Smith and Harry Tucker for their interest in my work and their 
support during the COVID-19 lockdowns. 
}

\begin{document}

\section{Introduction}

\subsection{Context}
Creating and rendering realistic models of trees manually requires advanced expertise 
with modelling software packages. This limits the ability to produce convincing 3D 
scenes for small developers with restricted resources.

The purpose of including trees in a natural environment is to provide realism. 
Trees are common natural structures present in even simple environments 
throughout the history of computer graphics and have seen many iterations as 
technology has advanced allowing for more detailed and realistic approaches.

The aim of this project is to provide a method for creating and rendering trees 
to be used in a real-time graphics application. This method should be simple to 
use and implement into an existing OpenGL project.

\subsection{Related Work}


\bibliography{bibfile}
\end{document}