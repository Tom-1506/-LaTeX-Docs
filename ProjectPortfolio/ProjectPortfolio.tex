\documentclass[final]{cmpreport}
\graphicspath{ {./images/} }

\title{GPU Accelerated Method for Constructing and Rendering Trees}
\author{Thomas Mcloughlin}
\date{21/3/2021}
\registration{100203952}
\ccode{CMP - 6013Y}
\supervisor{Dr. Stephen Laycock}

\summary
{
This project aims to convert and extend the Lindenmayer-system based tree construction 
method presented by \citep{prusinkiewicz1996systems} to be used as an independent 
OpenGL module. 
The module should allow the addition of trees to a real-time environment with minimal 
user interaction, avoiding the difficulties and expenses of manually producing tree 
models.
}

\acknowledgements
{
I'd like to thank my supervisor, Stephen Laycock for his brilliant help and support 
throughout this project.
Thank you to George Smith and Harry Tucker for their interest in my work and their 
support during the COVID-19 lockdowns. 
}

\begin{document}

\section{Introduction}

\subsection{Context}
Creating and rendering realistic models of trees manually requires advanced expertise 
with modelling software packages. This limits the ability to produce convincing 3D 
scenes for small developers with restricted resources.

The purpose of including trees in a natural environment is to provide realism. 
Trees are common natural structures present in even simple environments 
throughout the history of computer graphics and have seen many iterations as 
technology has advanced allowing for more detailed and realistic approaches.

The aim of this project is to provide a method for creating and rendering trees 
to be used in a real-time graphics application. This method should be simple to 
use and implement into an existing OpenGL project.

\subsection{Related Work}
In this section various related works will be discussed with respect to how they 
contribute to the main knowledge required for this project. These areas of main 
knowledge are the branching structure, branch thickness and leaf placement of the 
constructed trees.

A paper by \cite{lindenmayer1968mathematical} proposes a theory for the development 
of organisms using a cells current state and being combined with input that the cell 
receives from its neighbour. Two new cells are produced from this development to replace 
the existing cell and the cycle repeats for the two new cells. 

This process of generating new outputs recursively to produce larger structures became
``Lindenmayer systems'' or the abbreviated ``L-systems'' which became important tools 
in pattern generation for future applications. L-systems became a common approach for the 
generation of branching patterns in flora which is where the aim of this project is 
concerned.

Another early paper that is referenced by many later studies of the subject is by
\cite{honda1971description} where he presented one of the earliest algorithmic 
methods for creating a branching structure. This was done by starting with a parent 
branch which then bifurcates into two child branches and rotating them by given 
angles about the termination point of the parent branch, an example of which 
can be seen in Figure \ref{fig:honda-bifurcation}. By continuing this 
method they treated each child branch as another parent and bifuracted them to 
expand the branching structure which would be continued until a desired result is 
reached.

\begin{figure}[h]
    \caption{Example diagram of parent branch $P_AP_B$ bifurcating into child branches
    $P_BP_1$ with a rotation of $\theta_1$ and $P_BP_2$ with a rotation of \textminus$\theta_2$}
    \includegraphics[scale=0.5]{honda-bifurcation.PNG}
    \label{fig:honda-bifurcation}
    \centering
\end{figure}

This paper does not include any information relating to branch thickness or leaf 
placement as its focus was only on the branching structure. However, this paper 
provides a groundwork for many later studies that will be discussed in this project.



\bibliography{bibfile}
\end{document}